\documentclass[12.0pt, letterpaper]{article}
\usepackage[margin = 1in]{geometry}
\usepackage{amsmath}
\usepackage{amssymb}
\usepackage{fancyhdr}
\usepackage{pgfplots}
\pgfplotsset{compat=1.16}

\author{Sub}
\title{Test Homework}
\pagestyle{fancy}
\renewcommand{\headrulewidth}{0pt}

\newcommand{\biconditional}{\leftrightarrow}
\newcommand{\leftIndent}{\hspace*{8mm}}

\fancyhf{}
\rhead{
    Subham Sahoo\\
    Teacher: Professor Pelayo\\
    Math 0\\
}
\rfoot{Page \thepage}

\begin{document}

    \begin{center}
        \huge \sc Lesson 2 Solutions
    \end{center}

    \begin{paragraph}{1)}
    In the notes, we proved one of DeMorgan's Set Theory laws.  Prove the remaining one.  That is, prove the following statement:
    
    \begin{itemize}

        \item[]  Let $S$ and $T$ be sets.  Then $$\overline{S \cap T} = \overline S \cup \overline T.$$
        
    \end{itemize}
    \separate\spacing

    \textbf{Proposition:} $\overline{S \cap T} = \overline{S} \cup \overline{T}$ \spacing 

    \textbf{Discussion:} 
    \spacing

    \textit{What we want:} To prove the proposition we will prove two subset inclusions: 
    $\overline{S \cap T} \subset \overline{S} \cup \overline{T}$ and
    $\overline{S} \cup \overline{T} \subset \overline{S \cap T}$
    \spacing
    
    \textit{What we'll do:} Both subset inclusions will be proven
    using one of DeMorgan's logic laws that states 
    $\neg(p \land q) = \neg p \lor \neg q$ 
    \spacing

    \textbf{Proof:} To prove that $\overline{S \cap T} = \overline{S} \cup \overline{T}$,
    we need to prove the subset inclusions 
    $\overline{S \cap T} \subset \overline{S} \cup \overline{T}$ and
    $\overline{S} \cup \overline{T} \subset \overline{S \cap T}$.
    \spacing
    
    To prove the first inclusion, we will assume $x \in \overline{S \cap T}.$
    This can be rewritten as $x \notin S \cap T$ which means that it's 
    not true that $x \in S$ and $x \in T$. Using Demorgan's Logic Laws,
    this is logically equivlent to $x \notin S$ or $x \notin T$. This
    can be rewritten as $x \in \overline{S}$ or $x \in \overline{T}$
    which shows that $x \in \overline{S} \cup \overline{T}$. Thus, 
    $\overline{S \cap T} \subset \overline{S} \cup \overline{T}$.
    \spacing
    
    To prove the second inclusion, we will assume $x \in \overline{S} \cup \overline{T}$.
    This means that $x \in \overline{S}$ or $x \in \overline{T}$ which can be rewritten as
    $x \notin S$ or $x \notin T$. This means that it's not true that $x \in S$ or $x \in T$.
    Using DeMorgan's logic laws, this is logically equivalent to $x \notin S$ and $x \notin T$.
    This can be rewritten as $x \in \overline{S}$ and $x \in \overline{T}$, and since $x$
    is in the intersection, $x \in \overline{S \cap T}$. Thus, 
    $\overline{S} \cup \overline{T} \subset \overline{S \cap T}$. 
    \spacing

    Knowing both of these subset inclusions to be true, we know that our
    two sets are equal: $\overline{S \cap T} = \overline{S} \cup \overline{T}$.

    \proofEnd

    \end{paragraph}

    \bigskip

    \begin{paragraph}{2)}
        In the notes, we proved one distributive law.  Prove the remaining one.  That is, prove the following statement:

        \begin{itemize}

        \item[] Let $S, T,$ and $R$ be sets.  Then, $$S \cap (T \cup R) = (S \cap T) \cup (S \cap R).$$

        \end{itemize}

        \separate\spacing

        \textbf{Proposition:} $S \cap (T \cup R) = (S \cap T) \cup (S \cap R)$
        \spacing

        \textbf{Discussion:} 

        \textit{What we want:} To prove the proposition by proving
        two subset inclusions:
        $S \cap (T \cup R) \subset (S \cap T) \cup (S \cap R)$ and
        $(S \cap T) \cup (S \cap R) \subset S \cap (T \cup R$)
        \spacing

        \textit{What we'll do:} To prove the first inclusion $S \cap (T \cup R) \subset (S \cap T) \cup (S \cap R)$,
        we will assume that $x \in S \cap (T \cup R)$. This means we 
        can assume that $x \in S$ and $x \in (T \cup R)$ to be true.
        Since both parts must be true for the hypothesis to be true,
        at least one part of the union portion must be true i.e. either
        $x \in T$ or $x \in R$. Combining this with the fact that $x \in S$,
        we will show that $x \in (S \cap T)$ or $x \in (S \cap R)$.

        To prove the second inclusion $(S \cap T) \cup (S \cap R) \subset S \cap (T \cup R)$,
        we will assume that $x \in (S \cap T) \cup (S \cap R)$. Since this
        is an union assumption, we will look at two cases: 
        $x \in S \cap T$ and $x \in S \cap R$. In the first case, we'll show that 
        $x \in T$ naturally results in $x \in T \subset T \cup R$, therefore,
        $x \in (T \cup R)$. Since $x \in S$, we can combine these two to yield the 
        inclusion. A very similar process can be taken for the second case.     
        \spacing

        \textbf{Proof:} To prove that $S \cap (T \cup R) = (S \cap T) \cup (S \cap R)$,
        we will prove two subset inclusions: 
        $S \cap (T \cup R) \subset (S \cap T) \cup (S \cap R)$ and 
        $(S \cap T) \cup (S \cap R) \subset S \cap (T \cup R)$.
        \spacing

        For the first inclusion, we will assume $x \in S \cap (T \cup R)$.
        Thus, we know that $x \in S$ and $x \in T \cup R$. For the hypothesis to be
        true, each part of it must be true. Since the second part of the statement is a
        union, two cases arise: $x \in T$ or $x \in R$. Combining this with the first
        part of the hypothesis, we get the following:
        
        \begin{itemize}
            \item{
                \textbf{$x \in S$ and $x \in T$}: In this case, we're assuming that $x \in T$.
                Since $x \in S$, and $x \in T$, it can be said that $x \in S \cap T$.
            }
            $$or$$

            \item{
                \textbf{$x \in S$ and $x \in R$}: In this case, we're assuming that $x \in R$.
                With a similar logic to the previous bullet point, it can be shown that $x \in S \cap R$.
            }
        \end{itemize}

        Since both cases started by taking the possibilities of a union, the results of the 
        cases can be joined with a union resulting in $x \in (S \cap T) \cup (S \cap R)$.
        This proves the first inclusion that $S \cap (T \cup R) \subset (S \cap T) \cup (S \cap R)$.
        \spacing

        For the second inclusion, we will assume $x \in (S \cap T) \cup (S \cap R)$.
        Since this is a union hypothesis, we will look at two cases: 
        
        \begin{itemize}
            \item{
                $x \in (S \cap T)$: Since $x \in (S \cap T)$, $x \in S$ and $x \in T$. This can be
                rewritten as $x \in S$ and $x \in T \subset (T \cup R)$. Since $x$ is in both sets, $x$
                is in the intersection of the two sets: $x \in S \cap (T \cup R)$. Thus,
                $(S \cap T) \cup (S \cap R) \subset S \cap (T \cup R)$.
            }

            \item{
                $x \in (S \cap R)$: Since $x \in (S \cap R)$, $x \in S$ and $x \in R$. This
                can be rewritten as $x \in S$ and $x \in R \subset (R \cup T)$. Since $x$ is in
                both sets, it is in the intersection of both sets: $x \in S \cap (R \cup T)$.
                Thus, $(S \cap T) \cup (S \cap R) \subset S \cap (R \cup T)$.
            }
        \end{itemize}

        In both cases, we obtained the same conclusion thus proving the inclusion 
        $(S \cap T) \cup (S \cap R) \subset S \cap (T \cup R)$.
        \spacing

        Since we have now shown both inclusions to be true, we can say the 
        sets are equal: $S \cap (T \cup R) = (S \cap T) \cup (S \cap R)$.

        \proofEnd
    \end{paragraph}

    \bigskip

    \begin{paragraph}{3)}
        Let $A,B,C$, and $D$ be sets.  Show that if $A \subset B$ and $C \subset D$, then $A \times C \subset B \times D$.  

        \spacing\separate\spacing

        \textbf{Proposition:} $A \subset B \cap C \subset D \Rightarrow A \times C \subset B \times D$
        \spacing

        \textbf{Discussion:} We can start by defining $(x, y) \in A \times C$.
        Since $x \in A$ and $A \subset B$, $x \in B$.
        Similarly, since $y \in C$ and $C \subset D$, $y \in D$.
        Thus, $(x, y) \in B \times D$. This shows that $A \times C \subset B \times D$.
        \spacing

        \textbf{Proof:} To prove that $A \times C \subset B \times D$, let's
        start by supposing that $(x, y) \in A \times C$. This means that 
        $x \in A$ and $y \in C$. Since $x \in A$ and $A \subset B$,
        $x \in B$. Similarly, since $y \in C$ and $C \subset D$, $y \in D$.
        Thus, we can now state that $(x, y) \in B \times D$. Since we started with an 
        element $(x, y)$ in $A \times C$ and showed that it is also in $B \times D$,
        we have shown that $A \times C \subset B \times D$. Thus, we have proven that
        ``If $A \subset B$ and $C \subset D$, then $A \times C \subset B \times D$''. 

        \proofEnd

    \end{paragraph}

    \bigskip

    \begin{paragraph}{4)}
        Let $A, B$, and $C$ be sets.  Show that $$A \times (B \cap C) = (A \times B) \cap (A \times C).$$

        \separate\spacing

        \textbf{Proposition:} $A \times (B \cap C) = (A \times B) \cap (A \times C)$
        \spacing

        \textbf{Discussion:} 
        \spacing

        \textit{What we want:} We will prove the proposition by proving two subset inclusions:
        $$A \times (B \cap C) \subset (A \times B) \cap (A \times C)$$ 
        \begin{center}
            and
        \end{center}  
        $$(A \times B) \cap (A \times C) \subset A \times (B \cap C)$$

        \textit{What we'll do:} To prove the first inclusion, we'll take 
        $(x, y) \in A \times (B \cap C)$. This gives us that $x \in A$
        and $y \in (B \cap C)$. Since $y \in B$ and $y \in C$, we can combine this 
        $x \in A$ to get that $(x, y) \in A \times B$ and $(x, y) \in A \times C$.
        This can then be further combined to prove the first inclusion.
        \spacing

        To prove the second inclusion, we'll take $(x, y) \in (A \times B) \cap (A \times C).$
        This means $(x, y) \in A \times B$ which then yields $x \in A$ and $y \in B$.
        Similarly, $(x, y) \in A \times C$ which then yields $x \in A$ and $y \in C$.
        Since $y$ is in both $B$ and $C$, we can state that $y \in B \cap C$.
        Now we can see that $(x, y) \in A \times (B \cap C)$ which proves the 
        second inclusion.
        \spacing

        \textbf{Proof:} To prove $A \times (B \cap C) = (A \times B) \cap (A \times C)$,
        we need to prove two subset inclusions:
        $$A \times (B \cap C) \subset (A \times B) \cap (A \times C)$$
        \begin{center}and\end{center}
        $$(A \times B) \cap (A \times C) \subset A \times (B \cap C)$$

        To prove the first inclusion $A \times (B \cap C) \subset (A \times B) \cap (A \times C)$,
        let $(x, y) \in A \times (B \cap C)$. Thus, $x \in A$ and $y \in B \cap C$.
        Since $y \in B \cap C$, $y \in B$ and $y \in C$. Since $x \in A$ and $y \in B$,
        $(x, y) \in A \times B$. Similarly, since $x \in A$ and $y \in C$, 
        $(x, y) \in A \times C$. Since $(x, y)$ is in both sets, it is in the 
        intersection of the two sets such that $(x, y) \in (A \times B) \cap (A \times C)$.
        Thus, $A \times (B \cap C) \subset (A \times B) \cap (A \times C)$ proving the first
        subset inclusion.
        \spacing

        To prove the second inclusion $(A \times B) \cap (A \times C) \subset A \times (B \cap C)$,
        let $(x, y) \in (A \times B) \cap (A \times C)$. Thus, $(x, y) \in A \times B$
        which means $x \in A$ and $y \in B$. Similarly, $(x, y) \in A \times C$ which
        means $x \in A$ and $y \in C$. Since $y \in B$ and $y \in C$, $y \in B \cap C$.
        $x$ and $y$ can now be combined to write $(x, y) \in A \times (B \cap C)$.
        Thus, $(A \times B) \cap (A \times C) \subset A \times (B \cap C)$ proving the second
        subset inclusion.
        \spacing
        
        Now that we've proven the two subset inclusions, we can say the sets are equal:
        $A \times (B \cap C) = (A \times B) \cap \\(A \times C)$.
        \proofEnd
    \end{paragraph}

    \bigskip
    \pagebreak

    \begin{paragraph}{5)}
        Let $A, B$, and $C$ be sets. Show that $$A \times (B \cup C) = (A \times B) \cup (A \cup C)$$.

        \separate\spacing

        \textbf{Proposition:} $A \times (B \cup C) = (A \times B) \cup (A \times C)$
        \spacing

        \textbf{Discussion:} 
        \textit{What we want:} To prove the proposition by proving two subset inclusions:
        $$A \times (B \cup C) \subset (A \times B) \cup (A \times C)$$
        \begin{center} and \end{center}
        $$(A \times B) \cup (A \times C) \subset A \times (B \cup C)$$

        \textit{What we'll do:} To prove the first inclusion, we'll start by assuming that 
        $(x, y) \in A \times (B \cup C)$ which means $x \in A$ and $y \in B \cup C$. Since
        $y \in B \cup C$, $y \in B$ or $y \in C$. Combining this with $x \in A$, we get that
        $(x, y) \in A \times B$ or $(x, y) \in A \times C$. Combining these two statements yields
        the first inclusion. 
        \spacing

        To prove the second inclusion, assume that $(x, y) \in (A \times B) \cup (A \times C).$
        This means that $(x, y) \in A \times B$ or $(x, y) \in A \times C$. This can be broken down into
        $x \in A$ and $y \in B$ or $x \in A$ and $y \in C$. Using the distributive law $(A \cap B) \cup (A \cap C) = A \cap (B \cup C)$ we can rewrite
        the previous expressions as $x \in A$ and $y \in B \cup C$. From there, it's easy to see that 
        $(x, y) \in A \times (B \cup C)$ proving the second inclusion. 
        \spacing

        \textbf{Proof:} To prove the first inclusion $A \times (B \cup C) \subset (A \times B) \cup (A \times C)$,
        let $(x, y) \in A \times (B \cup C)$. This means $x \in A$ and $y \in B \cup C$. Since $y \in B \cup C$,
        $y \in B$ or $y \in C$. Each of the statements about $y$ can be combined with $x \in A$  to yield
        $(x, y) \in A \times B$ or $(x, y) \in A \times C$. This can be condensed as $(x, y) \in (A \times B) \cup (A \times C)$
        which proves the inclusion $A\times (B \cup C) \subset (A \times B) \cup (A \times C)$.
        \spacing

        To prove the second inclusion $(A \times B) \cup (A \times C) \subset A \times (B \cup C)$, let
        $(x, y) \in (A \times B) \cup (A \times C)$. Thus, $(x, y) \in A \times B$ or $(x, y) \in A \times C$.
        Put another way, $x \in A$ and $y \in B$ or $x \in A$ and $y \in C$. Using a distributive law,
        we can condense the expressions as $x \in A$ and $y \in B \cup C$. From there, we can combine the expressions
        into $(x, y) \in A \times (B \cup C)$. Thus, $(A \times B) \cup (A \times C) \subset A \times (B \cup C)$ 
        proving the second inclusion.
        \spacing

        Now that we've proven both inclusions, we have proven that the two sets are equal:
        $A \times (B \cup C) = (A \times B) \cup (A \times C)$.

        \proofEnd
    \end{paragraph}
\end{document}