\documentclass[10pt]{report}
\makeindex

\usepackage{amsmath, amsfonts, amssymb, amstext, amscd, amsthm, makeidx, graphicx, hyperref, url}
\allowdisplaybreaks

\setcounter{chapter}{0}
\topmargin -.5in
\textheight 9.0in

\begin{document}

\begin{center}\textsc{\Large Transition to Mathematical Proofs}

\textsc{\large Chapter 1 - Logic Assignment Solutions}

\end{center}

\bigskip


\noindent\textsc{Instructions:}  For the below questions, show all of your work.  For the proofs, be sure that you \\

\noindent(i) include a Discussion section; \\
(ii) write a complete proof in full English sentences; \\
(iii) if hand-writing, write legibly and clearly.


\bigskip


\noindent\textbf{Question 1.}  Let $m \neq 0$ and $b$ be real numbers.  Show that there exists a unique $x$ such that $mx+b=0$.  



\bigskip\bigskip

\noindent\textbf{Question 2.}  Prove the following biconditional statement. 


\begin{itemize}

\item[] Let $x$ be a real number.   $-1 \leq x \leq 1$ if and only if $x^2 \leq 1$.  

\end{itemize}

\noindent In proving this, it may be helpful to note that $-1\leq x \leq 1$ is equivalent to $ -1\leq x$ and $x \leq 1$.  

\bigskip\bigskip


\noindent\textbf{Question 3.} Two whole numbers are said to \emph{have the same parity} if they are both even or both odd.  Prove the following biconditional statement:

\begin{itemize}

\item[]Let $m$ and $n$ be whole numbers.  $m$ and $n$ have the same parity if and only if $m+n$ is even.  

\end{itemize}

\medskip



\bigskip


\noindent\textbf{Question 4.}  Use \emph{proof by contrapositive} to prove the following conditional statement.

\begin{itemize}

\item[] Let $m$ and $n$ be whole number.  If $m \cdot n$ is odd, then $m$ and $n$ are both odd.  

\end{itemize}

\medskip



\bigskip


\noindent\textbf{Question 5.}  We will investigate the following statement:

\begin{itemize}

\item[] Every odd whole number can be written as the difference of two perfect squares.  

\end{itemize}

\begin{itemize}

\item[(a)]  For the odd whole numbers $n = -3, -1, 1, 3, 5, 7, 9$, write $n$ as the difference of two perfect squares.

\item[(b)]  Use any pattern that you found in (a) to help you write a proof of our statement.  
\end{itemize}

\medskip


\end{document}