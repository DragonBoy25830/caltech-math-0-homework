\documentclass[12.0pt, letterpaper]{article}
\usepackage[margin = 1in]{geometry}
\usepackage{amsmath}
\usepackage{amssymb}
\usepackage{fancyhdr}
\usepackage{pgfplots}
\pgfplotsset{compat=1.16}

\author{Sub}
\title{Test Homework}
\pagestyle{fancy}
\renewcommand{\headrulewidth}{0pt}

\newcommand{\biconditional}{\leftrightarrow}
\newcommand{\leftIndent}{\hspace*{8mm}}

\fancyhf{}
\rhead{
    Subham Sahoo\\
    Teacher: Professor Pelayo\\
    Math 0\\
}
\rfoot{Page \thepage}

\begin{document}

	\begin{center}  

		\huge \sc Lesson 1 Solutions 
		
	\end{center}

	\begin{paragraph}{1)}
		Let $m \not = 0$ and b be real numbers. Show that there exists a unique $x$ such that $mx + b = 0$. \spacing

		\separate\spacing

		\textbf{Propsition:} $\exists!x$ $mx + b = 0$ \spacing
		
		\textbf{Discussion:} First we will show that there is an $x$ that solves $mx + b = 0$ by solving 
		the equation algebraically. Then we will show that it's the only 
		solution by assuming there are are two solutions, $x$ and $y$, 
		and show that $x=y$. \spacing
		
		\textbf{Proof.} Solving $mx + b = 0$ algebraically, first we subtract $b$ from each side to get $mx = -b$. 
		Diving $m$ by both sides, we get that $x = \frac{-b}{m}$. 
		Plugging this $x$ back into $mx + b = 0$ yields $m(\frac{-b}{m}) + b = -b + b = 0$, so we see this $x$ satisfies $mx + b = 0$.
		To show that this $x$ is unique, let's assume $x$ and $y$ both satisfy
		$mx + b = 0$. If $mx + b = 0$ and $my + b = 0$, then $mx + b = my + b$.
		Subtracting b from both sides, $mx = my$, and when we divide both sides by $m$, we get that
		$x = y$. Thus, there exists a unique $x$ that satisfies $mx + b = 0$. 

		\proofEnd
	
	\end{paragraph}

	\begin{paragraph}{2)}
		Prove the following biconditional statement.
 		\begin{center}	
			Let x be a real number. $-1 \leq x \leq 1$ if and only if $x^2 \leq 1$.
		\end{center}
		In proving this, it may be helpful to note that $-1 \leq x \leq 1$ is equivalent to $-1 \leq x$ and $x \leq 1$. \spacing

		\separate\spacing

		\textbf{Propsition:} $-1 \leq x \leq 1 \iff x^2 \leq 1$ \spacing
		
		\textbf{Discussion:} We'll break up this biconditional statement
		into $p \Rightarrow q:$ ``If $x$ is between $-1$ and $1$, then $x^2$ is 
		less than or equal to $1$", and $q \Rightarrow p:$ ``If $x^2$ is less
		than or equal to $1$, then $x$ is between $-1$ and $1$ (inclusive)." \spacing

		\leftIndent The first statement can be proven easily 
		by recognizing that p is a conjunction: $-1 \leq x \cap x \leq 1$.
		Since p is assumed to be true, each part of the conjuction is also 
		true. As such, we can work with each piece of the conjuction and
		prove that q is true for each piece, thus being true for p.
		Put another way, if we can show that $(-1 \leq x \Rightarrow x^2 \leq 1) 
		\cap (x \leq 1 \Rightarrow x^2 \leq 1)$, then we prove that 
		$p \Rightarrow q$. \spacing

		\leftIndent The second statement is a little trickier to prove
		since the hypothesis gives us information about $x^2$ and not $x$.
		The best way to get information about $x$ would be to use the contrapositive
		i.e. $\neg p \Rightarrow \neg q:$ ``If $x$ is not between $-1$ and $1$ (exclusive),
		then $x^2 > 1$." That is, we need to prove that $(x < -1 \cup
		x > 1) \Rightarrow x^2 > 1.$ Similar to the first statement, 
		since the hypothesis is a disjunction assumed to be true, each
		part of the disjunction can also be assumed to be true. We can show that $\neg q$ is true
		for each part of the disjunction, which proves $\neg p \Rightarrow \neg q$, 
		proving the second statement by proving the contrapositive. \spacing


		\textbf{Proof:} To prove the biconditional statement, we'll prove 
		two conditional statements \spacing

		\leftIndent The first statement $p \Rightarrow q$: ``If $x$ is between $-1$ and $1$, then $x^2$ is 
		less than or equal to $1$", can be proven by proving that q is true for each part of 
		the conjunction ($-1 \leq x \cap x \leq 1$) in the hypothesis. For the first part of the conjuction,
		we assume that $-1 \leq x$. Mulipling each side of the 
		inequality by $-1$ flips the inequality to $-x \leq 1$.
		Multiplying $-x \leq 1$ by itself won't change the sign and will instead yield
		$-x * -x \leq 1 * 1$ which is equivalent to $x^2 \leq 1$. For the second part of the
		conjunction, we assume that $1 \leq x$. Multiplying the inequality by itself won't change the sign
		and yields $1 * 1 \leq x * x$ which is equivalent to $x^2 \leq 1$.
		This proves the first statement.\spacing

		\leftIndent The second statement, $q \Rightarrow p:$ ``If $x^2$ is less
		than or equal to $1$, then $x$ is between $-1$ and $1$ (inclusive)", can be proven
		by proving the contrapositive, $\neg p \Rightarrow \neg q:$ ``If $x$ is not between $-1$ and $1$ (exclusive),
		then $x^2 > 1$." Put another way, we can prove that $(x < -1 \cup
		x > 1) \Rightarrow x^2 > 1$ by applying the same logic we did to the first statement i.e. proving 
		$\neg q$ to be true with each part of the disjunction. For the first part of the disjunction, 
		we assume that $x < -1$. Multipling each side of the inequality by $-1$
		flips the inequality to $-x > 1$. Multipling $-x > 1$ by itself won't change
		the sign of the inequality and yields $-x * -x > 1 * 1$ which is equivalent to $x^2 > 1$.
		For the second part of the disjunction, we assume that $x > 1$. Multiplying this inequality by 
		itself yields $x * x > 1 * 1$ which is equivalent to $x^2 > 1$. This proves the contrapositive
		$\neg p \Rightarrow \neg q$ which proves $q \Rightarrow p$. \spacing

		Since both $p \Rightarrow q$ and $q \Rightarrow p$ are proven to be true, the biconditinal statement
		``$-1 \leq x \leq 1$ if and only if $x^2 \leq 1$" is true. 
		
		\proofEnd

	\end{paragraph}

	\begin{paragraph}{3)}
		Two whole numbers are said to have the same parity if they are both even or both odd. Prove the following biconditional statement:
		\begin{center}
			Let m and n be whole numbers. $m$ and $n$ have the same parity if and only if $m + n$ is even.
		\end{center}

		\separate\spacing

		\textbf{Proposition:} $m$ and $n$ have the same parity $\iff (m + n) \equiv 0$ (mod $2$)\spacing

		\textbf{Discussion:} To prove the biconditional statement, we need to prove the two 
		statements $p \Rightarrow q:$ ``If $m$ and $n$ have the same parity, then $m + n$ is even."
		and $q \Rightarrow p:$ ``If $m + n$ is even, then $m$ and $n$ have the same parity.'' \spacing

		\leftIndent The first statement is relatively easy to prove because
		we're already assuming that $m$ and $n$ have the same parity. We can look 
		at two cases: $m$ and $n$ are both even, or $m$ and $n$ are both odd.
		If $m$ and $n$ are both even, then we can represent them as 
		$2k$ and $2l$, respectively, where $k \in \N$ and $l \in \N$. 
		Adding $2k$ and $2l$ together yields $2k + 2l = 2(k + l)$ which is divisible by $2$, 
		therefore making the sum even. If $m$ and $n$ are both odd, then we can represent them as
		$2k + 1$ and $2l + 1$, respectively, where $k \in \N$ and
		$l \in \N$. Adding $2k + 1$ and $2l + 1$ yields 
		$2k + 2l + 2$ which is divisible by $2$, therefore making the sum even.
		In both cases where $m$ and $n$ had the same parity, $m + n$
		was even, therefore proving $p \Rightarrow q$. \spacing

		\leftIndent The second statement is a little tricker to prove in its current form
		because its difficult to extract information about $m$ and $n$ from
		$m + n$, so it'd be useful to prove the contrapositive instead.
		The contrapositive is $\neg p \Rightarrow \neg q:$ ``If $m$ and $n$
		don't have the same parity, then $m + n$ is odd''. This is much easier 
		to prove. We can represent $m$ and $n$ as $2k$ and $2l + 1$ where
		$k \in \N$ and $l \in \N$. The order of this assignment doens't matter,
		as long as both numbers have different parities, because addition is
		commutative. Adding $2k$ and $2l + 1$ yields $2k + 2l + 1$ which will always be odd.
		This proves the contrapositive which then proves $q \Rightarrow p$. \spacing

		\leftIndent Since we proved both conditional statements, the 
		biconditional statement ``If $m$ and $n$ have the same parity,
		then $m + n$ is even'' is proven true. \spacing

		\textbf{Proof:} To prove the biconditional statement, we will prove two conditional statements.\spacing

		\leftIndent The first statement $p \Rightarrow q:$ ``If $m$ and $n$ have the same parity, then $m + n$ is even"
		can be proven by analyzing the two cases that arise from assuming the hypothesis to be true. 
		If $m$ and $n$ have the same parity, then either $m$ and $n$ are both odd or both even. \

		\begin{itemize}
			\item{
				\textbf{$\mathbf{\textit{m}}$ and $\mathbf{\textit{n}}$ are even:} Since both are even, we can represent them as 
				$2k$ and $2l$, respectively, where $k \in \N$ and $l \in \N$. Adding these two
				expressions yields $m + n = 2k + 2l = 2(k+l)$. No matter what $k$ and $l$ are, 
				multiplying $k + l$ by $2$ makes it divisible by two 
				($\frac{m + n}{2} = \frac{2(k + l)}{2} = k + l \equiv 0$ (mod $2$)) and therefore even. 
				This proves that if $m$ and $n$ are even, then $m + n$ is even.
			}
		
			\item{
				\textbf{$\mathbf{\textit{m}}$ and $\mathbf{\textit{n}}$ are odd:} If $m$ and
				$n$ are odd, then they can be represented as $2k + 1$ and $2l + 1$, respectively,
				where $k \in \N$ and $l \in \N$. Adding these two expressions yields 
				$m + n = 2k + 1 + 2l + 1 = 2k + 2l + 2 = 2(k + l + 1)$. If we divide the sum by $2$,
				$\frac{m + n}{2} = \frac{2(k + l + 1)}{2} = k + l + 1 \equiv 0$ (mod $2$), then we can see that is is divisible
				by $2$ and therefore even. This proves that when  $m$ and $n$ are odd, then $m + n$ is even.
			}
		\end{itemize}

		\begin{flushleft}
			Now that we've proved both cases, we can say that the first statement $p \Rightarrow q$ is true.
		\end{flushleft}

		\leftIndent The second statement $q \Rightarrow p$ will be proven by proving the contrapositive statement
		$\neg p \Rightarrow \neg q:$ ``If $m$ and $n$ don't have the same parity, then $m + n$ will be odd.'
		Since $m$ and $n$ are of different parities, one of them is odd and one of them is even. That means that 
		they can be represented as $2k$ and $2l + 1$, where $k \in \N$ and $l \in \N$, in no particular order because 
		addition is commutative (and it doesn't matter whether $m$ or $n$ is odd). Adding these two yields
		$m + n = 2k + 2l + 1$. When diving this sum by 2, $\frac{m + n}{2} = \frac{2k + 2l + 1}{2} \equiv 1$ (mod $2$),
		it's clear that the sum is not divisible by two and therefore odd. This proves the contrapositive $\neg p \Rightarrow \neg q$
		which then proves $q \Rightarrow p$. \spacing
		
		Now that we have proved that $p \Rightarrow q$ and $q \Rightarrow p$, we have proved the biconditional statement
		$p \iff q$ that ``$m$ and $n$ have the same parity if and only if $m + n$ is even.'' 
		
		\proofEnd
	
	\end{paragraph}

	\begin{paragraph}{4)}
		Use proof by contrapositive to prove the following conditional statement.
 		\begin{center}
			Let $m$ and $n$ be whole numbers. If $m * n$ is odd, then $m$ and $n$ are both odd.
		\end{center}
		
		\separate\spacing

		\textbf{Proposition:} $m * n \equiv 1$ (mod $2$) $\Rightarrow m \equiv 1$ (mod $2$) $\cap$ $n \equiv 1$ (mod $2$) \spacing

		\textbf{Discussion:} Since our hypothesis gives us information about $m * n$ and not $m$ or $n$, it would 
		be useful to work with the contrapositive $\neg q \Rightarrow \neg p$ which states ``If either $m$ or $n$ is even,
		then $m * n$ is even.'' This statement is much easier to prove because if we're assuming one of the numbers to be even,
		then it can be written as $2k$ where $k \in \N$. Irregardless of $n$'s parity, the resulting product will be divisible by two
		and therefore even. Proving the contrapositive proves the original statement. 
		 
		\textit{We will assume:} $m$ or $n$ is even
		 
		\textit{We will prove:} $m * n$ is even. \spacing

		\textbf{Proof:} To prove the statement, we will prove the contrapositive $\neg q \Rightarrow \neg p$
		that states ``If either $m$ or $n$ is even, then $m * n$ is even.'' We're assuming that at 
		least one of the numbers is even, so let's take $m$ to be even for now. If $m$ is even, then
		it can be represented as $2k$ where $k \in \N$. Multiplying $m$ with $n$ yields 
		$m * n = 2k * n = 2 * (k * n)$. Dividing this expression with 2, $\frac{m * n}{2} = 
		\frac{2 * (k * n)}{2} = k * n \equiv 0$ (mod $2$), shows that not only do
		we get a product that is divisible by $2$ and therefore even, but that the parity of $n$ does not matter.
		This process can recreated again if we take just $n$ to be even or both $m$ and $n$ to be even. 
		This proves the contrapositive $\neg q \Rightarrow \neg p$ which then proves the original 
		statement that ``If $m * n$ is odd, then $m$ and $n$ are both odd.''

		\proofEnd

	\end{paragraph}

	\begin{paragraph}{5)}
		We will investigate the following statement:
		
		\begin{center}
			Every odd whole number can be written as the difference of two perfect
			squares.
		\end{center}

		(a)  For the odd whole numbers n = -3,-1,1,3,5,7,9, write n as the difference of two perfect squares. \spacing
		
		(b)  Use any pattern that you found in (a) to help you write a proof of our statement.
	
		\separate\spacing

		\textbf{a)}
		\begin{center}
		\begin{tabular}{||c | c||}
			
			\hline
			n & Difference of squares \\[0.5ex]
			\hline\hline
			-$3$ & $1^2 - (-2)^2$ \\
			\hline 
			-$1$ & $0^2 - (-1)^2$ \\
			\hline 
			$1$ & $1^2 - 0^2$ \\
			\hline 
			$3$ & $2^2 - 1^2$ \\
			\hline 
			$5$ & $3^2 - 2^2$ \\
			\hline 
			$7$ & $4^2 - 3^2$ \\
			\hline 
			$9$ & $5^2 - 4^2$ \\
			\hline 
		\end{tabular}
	\end{center}
	\end{paragraph}

	\pagebreak

	\textbf{b) Proposition:} Every odd whole number can be written as the 
	difference of two perfect squares. \spacing

	\textbf{Discussion:} After filling out the table with a few
	odd numbers and their corresponding difference of perfect squares, I
	notices that all the perfect squares were created from consecutive whole
	numbers. Even the negative odd numbers continue this progression if you
	rewrite the second number as a negative (gets turned positive when squaring it).
	This pattern is what I honed in on to show that every odd number can 
	be written as a difference of perfect squares, and I will show this pattern 
	by demonstrating how $2k + 1$, where $k \in \N$, can be rewritten algebraically as a difference
	of perfect squares. \spacing
	
	\textit{We will assume:} We're starting an odd whole number

	\textit{We will prove:} Every odd whole number can be expressed as a difference of perfect squares \spacing

	\textbf{Proof:} Since we're starting with an odd whole number, we can express 
	that as $2k + 1$ where $k \in \N$. The expression can be rewritten as follows:
	$2k + 1 = 2k + 1 + k^2 - k^2$. After reordering the terms a bit, we get
	$2k + 1 = k^2 + 2k + 1 - k^2$. The first three terms on the right side 
	of the equation are the expanded form of the square of a binomial, so it can 
	be condensed to $2k + 1 = (k + 1)^2 - k^2$. From this form, it becomes evident
	that $2k + 1$ (the odd number) is written as the difference between the square 
	of $k + 1$ and $k$. This proves that every odd whole number can be written as the 
	difference between two perfect squares.

	\proofEnd
\end{document}