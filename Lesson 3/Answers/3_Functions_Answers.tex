\documentclass[12.0pt, letterpaper]{article}
\usepackage[margin = 1in]{geometry}
\usepackage{amsmath}
\usepackage{amssymb}
\usepackage{fancyhdr}
\usepackage{pgfplots}
\pgfplotsset{compat=1.16}

\author{Sub}
\title{Test Homework}
\pagestyle{fancy}
\renewcommand{\headrulewidth}{0pt}

\newcommand{\biconditional}{\leftrightarrow}
\newcommand{\leftIndent}{\hspace*{8mm}}

\fancyhf{}
\rhead{
    Subham Sahoo\\
    Teacher: Professor Pelayo\\
    Math 0\\
}
\rfoot{Page \thepage}

\begin{document}

    \begin{center}
        \huge \sc Lesson 3 Solutions
    \end{center}

    \begin{paragraph}{1)}
    
    \textbf{a) Proposition:} Assuming $m \neq 0$ and $b$ are real numbers, when $f: \R \rightarrow \R$ given by $f(x) = mx + b$, $f$ is a bijection.  
    \spacing

    \textbf{Dicussion:} To prove that $f$ is a bijection, we need
    to prove that $f$ is both an injection and a surjection. 
    \spacing

    To prove that $f$ is an injection, we will start by assuming that 
    $f(s_1) = f(s_2)$ and use algebraic manipulation on that expression 
    to show that $s_1 = s_2$. 
    \spacing

    To prove that $f$ is surjective, we will start by letting $y \in \R$.
    We will set $f(x) = y$ and solve for $x$. Then, we will show that $x \in \R$
    thus making $f$ a surjection. 
    \spacing

    \textbf{Proof:} To prove that $f$ is a bijection, we need to prove that $f$ is an injection and a surjection.
    \spacing

    For injectivity, we will let $f(s_1) = f(s_2)$ and show that $s_1 = s_2$.
    Since $f(s_1) = f(s_2)$, we can say that $m(s_1) + b = m(s_2) + b$.
    Subtracting $b$ from both sides, we get $m(s_1) = m(s_2)$, and dividing $m$ from both sides,
    we get $s_1 = s_2$. When $f(s_1) = f(s_2)$, $s_1 = s_2$, so $f$ is an injection.
    \spacing 

    For surjectivity, let $y \in \R$. Let's start by considering $f(x) = y$ and solve for $x$.
    $$f(x) = y$$
    $$mx + b = y$$
    $$mx = y - b$$
    $$x = \frac{y-b}{m}$$
    
    Since $x = \frac{y - b}{m} \in \R$, $f$ is surjective. 
    In other words, for $y \in \R$ we have shown that $x$ is the pre-image of $y$ ($f^{-1}(y) = x$),
    thus proving that $f$ is surjective.
    \spacing

    Now that we've proven that $f$ is injective and surjective,
    we have proven that $f$ is a bijection.
    \bigskip

    \textbf{b) Proposition:} $f$ being a bijection makes it invertible, so find $f^{-1}$ and show that it is an inverse.
    \spacing

    \textbf{Discussion:} Since $f$ is a bijection, we can find its' inverse
    $f^{-1}$ by inverting the domain and co-domain and show that $f^{-1}$ is 
    an inverse by demonstrating that $f^{-1}(f(x)) = x$.
    \spacing

    \textbf{Proof:} Inverse functions invert the domain and co-domain,
    so $f^{-1}(x)$ can be found by solving $x = mf^{-1}(x) + b$. Subtracing $b$
    from both sides, we get that $x - b = mf^{-1}(x)$, and dividing $m$ from 
    both sides gives us $f^{-1}(x) = \frac{x - b}{m}$.
    \spacing

    To show that $f^{-1}(x)$ is an inverse, we'll show that $f^{-1}(f(x)) = x$.
    $$f^{-1}(f(x)) = f^{-1}(mx + b) = \frac{(mx + b) - b}{m} = \frac{mx}{m} = x$$.
    Thus, $f^{-1}(x)$ is an inverse.
    \proofEnd
    \end{paragraph}
    
    \bigskip

    \begin{paragraph}{2)}    
    \textbf{Proposition:} Given that $\gamma, \rho \in \R$ such that $\rho \cdot \gamma \neq 1$
    and $f: \R - \{-\rho\} \rightarrow \R - \{\gamma\}$, let $$f(x) = \frac{\gamma x + 1}{x + \rho}$$.
    
    Show that $f$ is a bijection.
    
    \spacing
    
    \textbf{Discussion:} To show that $f$ is a bijection, we need to show that $f$ is injective and surjective.

    To show $f$ is an injection, we will start by letting $f(s_1) = f(s_2)$.
    We will use algebraic manipulation of this expression to show that $s_1 = s_2$.
    \spacing

    
    To show that $f$ is surjective, let $y \in \R - \{\gamma\}$. We will solve $f(x) = y$ for $x$
    and show that $x \in \R - \{-\rho\}$ by showing how $x \neq -\rho$ (since $-\rho$ is the only
    value excluded from the domain, if we can show that $x \neq -\rho$, then $x$ must be in 
    the domain). Now, we have shown $x$ is a pre-image of $y$ and therefore $f$ is surjective.
    \spacing

    \textbf{Proof:} To prove that $f$ is a bijection, we need to show that $f$ is an injection and surjection.
    \spacing

    To show injectivity, we will let $f(s_1) = f(s_2)$ and show that $s_1 = s_2$.
    Let's work with $f(s_1) = f(s_2)$.
    $$f(s_1) = f(s_2)$$
    $$\frac{\gamma s_1 + 1}{s_1 + \rho} = \frac{\gamma s_2 + 1}{s_2 + \rho}$$
    $$(\gamma s_1 + 1)(s_2 + \rho) = (\gamma s_2 + 1)(s_1 + \rho)$$
    $$\gamma s_1 s_2 + \gamma\rho s_1 + s_2 + \rho = \gamma s_1 s_2 + \gamma\rho s_2 + s_1 + \rho$$
    $$\gamma\rho s_1 + s_2 = \gamma\rho s_2 + s_1$$
    $$\gamma\rho s_1 - s_1 = \gamma\rho s_2 - s_2$$
    $$s_1 (\gamma\rho - 1) = s_2 (\gamma\rho - 1)$$
    $$s_1 = s_2$$

    We have shown that when $f(s_1) = f(s_2)$, $s_1 = s_2$. Thus,
    $f$ is injective.
    \spacing

    To show $f$ is a surjection, we will let $y \in \R - \{\gamma\}$.
    Let's solve $f(x) = y$ for $x$.
    $$f(x) = y$$
    $$\frac{\gamma x + 1}{x + \rho} = y$$
    $$\gamma x + 1 = y(x + \rho)$$
    $$\gamma x + 1 = xy + \rho y$$
    $$\gamma x - xy = \rho y - 1$$
    $$x (\gamma - y) = \rho y - 1$$
    $$x = \frac{\rho y - 1}{\gamma - y}$$

    To show that $x \in \R - \{-\rho\}$, we will show that $x \neq -\rho$. We'll start by 
    assuming $x = -\rho$ and show how a contradition arises.
    $$x = -\rho$$
    $$\frac{\rho y - 1}{\gamma - y} = -\rho$$
    $$\rho y - 1 = -\rho(\gamma - y)$$
    $$\rho y - 1 = \rho y - \rho\gamma $$
    $$-1 = -\rho\gamma$$
    $$\rho\gamma = 1$$
    This contradicts the original given condition that $\rho\gamma \neq 1$, so our 
    original statement, $x = -\rho$ is false. Thus, $x \neq -\rho$ and therefore
    $x \in \R - \{-\rho\}$.
    \spacing

    We have shown that $x$ is a pre-image of $y$ (both $x$ and $y$ are within
    the domain and range of $f$, respectively) such that $f(x) = y$.
    Thus, $f$ is surjective.
    \spacing

    Since $f$ is injective and surjective, we have shown that $f$ is a bijection.
    \proofEnd
    \end{paragraph}

    \bigskip

    \begin{paragraph}{3)}
    \textbf{Proposition:} Let $S, T$, and $R$ be sets, and let $f: S \to T$ and $g: T \to R$ be functions.  If $g \circ f$ is injective, then $f$ is injective.  
    \spacing

    \textbf{Discussion:}
    
    \textit{What we know:} $(g \circ f)$ is injective. Thus, if $(g \circ f)(s_1) = g \circ f(s_2)$, then we know that $s_1 = s_2$. 
    \spacing

    \textit{What we want:} To prove that $f$ is injective, we will need to show that when $f(s_1) = f(s_2)$, $s_1 = s_2$.
    \spacing

    \textit{What we'll do:} We'll start by looking at $g \circ f$ being injective.
    Since it is injective, when we input $(g \circ f)(s_1) = (g \circ f)(s_2)$, we 
    know that the inputs are equal: $s_1 = s_2$. Now, if we input $f(s_1)$ and $f(s_2)$,
    we get from the injectivity of $g \circ f$, $(g \circ f)(f(s_1) = (g \circ f)(s_2))$,
    that $f(s_1) = f(s_2)$. Thus we now have two equalities that when put together
    make $f$ injuctive by definition.  
    \spacing

    \textbf{Proof:} We will show that $f$ is injective by showing that $s_1 = s_2$, 
    for $s_1, s_2 \in S$, and $f(s_1) = f(s_2)$.
    \spacing
    
    We'll start by looking at $g \circ f$ being injective. Thus, when 
    $(g \circ f)(s_1) = (g \circ f)(s_2)$, we can say that $s_1 = s_2$.
    Similarly, when $(g \circ f)(f(s_1)) = (g \circ f)(f(s_2))$, we can say that
    $f(s_1) = f(s_2)$. Since $f(s_1) = f(s_2)$ and $s_1 = s_2$, we can say that 
    $f$ is injective. 

    \proofEnd
    \end{paragraph}

    \bigskip

    \begin{paragraph}{4)}
        Let $C([0,1])$ be the set  of all real, continuous functions on the interval $[0,1].$  That is, $$C([0,1]) = \left\{ f \, \vert \, f: [0,1] \to \mathbb R \textrm{   is a continuous function} \right\}.$$  Thus, an element of the set $C([0,1])$ is simply a function $f(x)$ that is continuous on $[0,1]$.   Furthermore, consider the function $\varphi: C([0,1]) \to \mathbb R$ given by $$\varphi (f) = \int_0^1 f(x) \, dx.$$

        \textbf{a) Proposition:} $\forall a \in \R$, there exists a pre-image $f \in C([0, 1]$) such that $\varphi(f) = a$, so $\varphi$ is surjective.
        \spacing

        \textbf{Discussion:} To prove that a general function is surjective, we need to start with some
        arbitrary $y \in Y$, solve for $f(x) = y$, and check whether $x \in X$. To prove $\varphi$ is surjective,
        we'll start by letting $a \in \R$. From there, we'll solve the equation $\varphi(f) = a$
        and present a valid solution for $f$. If $f \in C([0, 1])$, then we have proven the surjectity of $\varphi$.
        \spacing

        \textbf{Proof:} Let $a \in \R$. We'll start by considering $\varphi(f) = a$
        and solving for $f$. 
        $$\int_0^1 f(x) \, dx = a$$
        $$F(1) - F(0) = a$$ 
        \begin{center}
            where $F(x)$ is an antiderivative of $f(x)$
        \end{center}
        From this, we can see that one possible solution is $f(x) = a$.
        Since $f(x) = a \in C([0, 1])$, we know that the $\varphi$ is subjective
        since we started with $a \in \R$ and showed that there was a valid pre-image
        $f \in C([0, 1])$ such that $\varphi(f) = a$.
        
        \bigskip
        \bigskip
        
        \textbf{b) Proposition:} $\varphi$ is not injective
        \spacing
        
        \textbf{Discussion:} To prove that $\varphi$ is not injective,
        we need to find two distinct functions, $f, g \in C([0, 1])$, such that 
        $\varphi(f) = \varphi(g)$. Put another way, if $\varphi$ is injective, then
        when we start with $f \neq g$, $\varphi(f) \neq \varphi(g)$. If 
        $\varphi(f) = \varphi(g)$, then we know $\varphi$ is not injective.
        \spacing

        \textbf{Proof:} To prove that $\varphi$ is not injective,
        let $f(x) = 1$ and $g(x) = 2x$ such that $f, g \in C([0, 1])$. 
        Note that $f \neq g$. We'll evaulate $\varphi(f)$ and $\varphi(g)$.
        $$\varphi(f) = \int_0^1 1 \, dx = x |_0^1 = 1 - 0 = 1$$
        $$\varphi(g) = \int_0^1 2x \, dx = x^2 |_0^1 = 1 - 0 = 1$$.
        From this, we see that $\varphi(f) = \varphi(g)$ and since
        $f \neq g$, we have proven that $\varphi$ is not injective.

    \proofEnd
    \end{paragraph}

\end{document}