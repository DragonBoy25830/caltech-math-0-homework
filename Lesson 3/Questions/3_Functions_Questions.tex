\documentclass[10pt]{report}
\makeindex

\usepackage{amsmath, amsfonts, amssymb, amstext, amscd, amsthm, makeidx, graphicx, hyperref, url}
\allowdisplaybreaks

\setcounter{chapter}{0}
\topmargin -.5in
\textheight 9.0in

\begin{document}

\begin{center}\textsc{\Large Transition to Mathematical Proofs}

\textsc{\large Chapter 3 - Functions Assignment}

\bigskip

\end{center}

\noindent\textsc{Instructions:}  For the below questions, show all of your work.  For the proofs, be sure that you \\

\noindent(i) include a Discussion section; \\
(ii) write a complete proof in full English sentences; \\
(iii) if hand-writing, write legibly and clearly.


\bigskip

\noindent\textbf{Question 1.}  Let $m\neq 0$ and $b$ be real numbers and consider the function $f: \mathbb R \to \mathbb R$ given by $f(x) = mx + b$.  

\begin{itemize}

\item[(a)]  Prove that $f$ is a bijection.

\item[(b)]  Since $f$ is a bijection, it is invertible.  Find its inverse $f^{-1}$, and show it is an inverse by demonstrating that $$f^{-1}(f(x)) = x.$$


\end{itemize}

\bigskip\bigskip

\noindent\textbf{Question 2.}  Let $\gamma, \rho \in \mathbb R$ be real numbers such that $\gamma \cdot \rho \neq 1$.   Let $\mathbb R \! - \! \{\gamma\}$ and $\mathbb R \!- \! \{-\rho\}$ be the set of all real numbers $\mathbb R$ except for $\gamma$ and $-\rho,$ respectively.  Consider the function $f: \mathbb R \! - \! \{-\rho\} \to \mathbb R \! - \! \{\gamma\}$ given by $$f(x) = \frac{\gamma x + 1}{x + \rho}.$$  Show that $f$ is a bijection.  

\bigskip\bigskip


\noindent\textbf{Question 3.}  Let $S, T$, and $R$ be sets, and let $f: S \to T$ and $g: T \to R$ be functions.  Show that if $g \circ f$ is injective, then $f$ is injective. 



\bigskip\bigskip


\noindent\textbf{Question 4.}  Let $C([0,1])$ be the set  of all real, continuous functions on the interval $[0,1].$  That is, $$C([0,1]) = \left\{ f \, \vert \, f: [0,1] \to \mathbb R \textrm{   is a continuous function} \right\}.$$  Thus, an element of the set $C([0,1])$ is simply a function $f(x)$ that is continuous on $[0,1]$.   Furthermore, consider the function $\varphi: C([0,1]) \to \mathbb R$ given by $$\varphi (f) = \int_0^1 f(x) \, dx.$$

\begin{itemize}

\item[(a)]  Show that the function $\varphi$ is surjective by showing that for every $a \in \mathbb R$, there exists a pre-image $f \in C([0,1])$ such that $\varphi (f) = a$.  

\item[(b)]  Show that the function $\varphi$ is not injective by finding two distinct functions $f,g \in C([0,1])$ such that $\varphi(f) = \varphi(g)$.

\end{itemize}

\end{document}

