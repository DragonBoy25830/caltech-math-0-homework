\documentclass[12.0pt, letterpaper]{article}
\usepackage[margin = 1in]{geometry}
\usepackage{amsthm}
\usepackage{amsmath}
\usepackage{amssymb}
\usepackage{fancyhdr}
\usepackage{pgfplots}
\pgfplotsset{compat=1.16}

\author{Sub}
\title{Test Homework}
\pagestyle{fancy}
\renewcommand{\headrulewidth}{0pt}

\newcommand{\biconditional}{\leftrightarrow}
\newcommand{\leftIndent}{\hspace*{8mm}}
\newcommand{\spacing}{\vspace*{6.0pt}}
\newcommand{\N}{\mathbb{N}}
\newcommand{\Z}{\mathbb{Z}}
\newcommand{\Q}{\mathbb{Q}}
\newcommand{\R}{\mathbb{R}}
\newcommand{\C}{\mathbb{C}}
\newcommand{\separate}{\makebox[0.5\columnwidth]{\dotfill}}

\fancyhf{}
\rhead{
    Subham Sahoo\\
    Teacher: Professor Pelayo\\
    Math 0\\
}
\rfoot{Page \thepage}

% Absolutely goated reference: https://oeis.org/wiki/List_of_LaTeX_mathematical_symbols

\begin{document}
    \begin{center}
        \huge \sc Lesson 7 Solutions
    \end{center}

    \begin{paragraph}{1)}
        \textbf{Proposition:} Let $a, b, c \in \N$. Show that if $a + b = a + c$, then $b = c$.
        \spacing
        
        \textbf{Discussion:} We will induct on $a$ and keep $b$ and $c$ fixed, but arbitrary.
        \spacing

        First, we'll show the base case $a = 0$ to be true for the statement by using
        the definition and commutativity of addition.
        \spacing

        Then, we'll assume for $a \in \N$,
        $$a + b = a + c \Rightarrow b = c$$
        We will show that for $S(a) \in \N$,
        $$S(a) + b = S(a) + c \Rightarrow b + c$$

        \textbf{Proof:} To prove $p \Rightarrow q$: $a + b = a + c \Rightarrow b = c$, we will first show that the statement is true by inducting on $a$ to show that $p$ and $q$ are true.
        \spacing

        First, we'll show the base case $a = 0$ to be true. We'll start by assuming $p$ to be true and showing that 
        the statement holds.
        \begin{align*}
            a + b \stackrel{?}{=} a + c \\
            0 + b \stackrel{?}{=} 0 + c \\
            b + 0 \stackrel{?}{=} c + 0 && \text{(commutativity of addition)} \\
            b = c && \text{(definition of addition)}
        \end{align*}
        
        Thus, we have shown that the base case holds.
        \spacing

        Now, we will start with the inductive step and assume that for $a \in \N$,
        $$a + b = a + c \Rightarrow b = c$$
        We will show that this statement is true for $S(a)$ by showing that
        $$S(a) + b = S(a) + c \Rightarrow b = c$$

        Looking at the left side of the hypothesis of the $S(a)$ statement, we can show the following:
        \begin{align*}
            S(a) + b\\
            b + S(a) && \text{(commutativity of addition)} \\
            S(b + a)&& \text{(definition of addition)} \\
            S(a + b) && \text{(commutativity of addition} \\
            S(a + c) && \text{(hypothesis of inductive assumption)} \\
            S(c + a) && \text{(commutativity of addition)} \\
            c + S(a) && \text{(definition of addition)} \\
            S(a) + c && \text{(commutativity of addition)}
        \end{align*}
        Now, we have shown the hypothesis of the $S(a)$ statement to be true. The condition of the $S(a)$ statement is true as outlined by 
        the inductive assumption. Thus, we have used our inductive assumption to show that the $S(a)$ statment is true.
        \spacing
        
        By induction, we have proven that for $a, b, c \in \N$, if $a + b = a + c$, then $b = c$ \qed
    \end{paragraph}
    
    \bigskip

    \begin{paragraph}{2)}
        Let $a \in \N$
        
        \textbf{a) Proposition:} Show that $a + a = 2 \cdot a$ where $2$ is the symbol defined by $S(S(0))$.
        \spacing

        \textbf{Discussion:} We'll start this proof by inducting on $a$.
        \spacing

        First we'll show that the base case $a = 0$ is true using the definition of $+$ and $\times$.
        \spacing

        Then, we'll assume for $a \in \N$,
        $$a + a = 2 \cdot a$$
        We will show that for $S(a) \in \N$,
        $$S(a) + S(a) = 2 \cdot S(a)$$

        \textbf{Proof:} To prove the proposition, we'll use induction and induct on $a$.
        \spacing

        First, we'll show that the base case $a = 0$ holds true:
        \begin{align*}
            a + a \stackrel{?}{=} 2 \cdot a \\
            0 + 0 \stackrel{?}{=} 2 \cdot 0 \\
            0 \stackrel{?}{=} 2 \cdot 0 && \text{(definition of addition)}\\
            0 = 0 && \text{(definition of multiplication)}
        \end{align*}
        Thus, we have shown the base case $a = 0$ to be true by using axiom 1.
        \spacing

        Now, we'll start with the inductive step and assume for $a \in \N$,
        $$a + a = 2 \cdot a$$
        We will show that for $S(a) \in \N$,
        $$S(a) + S(a) = 2 \cdot S(a)$$

        Before we start working the the $S(a)$ statement, it's worthwhile to note that
        \begin{align}x + 1 = x + S(0) = S(x + 0) = S(x)\end{align}

        Looking at the left hand side of the $S(a)$ statement, we can show the following
        \begin{align*}
            S(a) + S(a) \\
            a + 1 + S(a) && \text{(by statement (1) above)} \\
            (a + 1) + S(a) \\
            S((a + 1) + a) && \text{(definition of addition)} \\
            S(a + a + 1) && \text{(commutativity of addition)} \\
            S(2 \cdot a + 1) && \text{(inductive assumption)} \\
            (2 \cdot a + 1) + 1 && \text{(by statement (1))} \\
            2\cdot a + 2 \\
            2 \cdot (a + 1) && \text{(distributivity theorem)} \\
            2 \cdot S(a) && \text{(by statement (1))} \\
        \end{align*}
        Thus, we have used our inductive assumption to show that the $S(a)$ statment is true.
        \spacing
        
        By induction, we now know that for $a \in \N$, $a + a = 2 \cdot a$. \qed
        \bigskip

        \textbf{b) Proposition:} For the $n$-fold sum $a + \cdots + a = n \cdot a$.
        \spacing

        \textbf{Discussion:} We'll start by inducting over $n$ and showing that 
        the base case $n = 0$ is true by using the definition and commutativity of multiplication.
        For ease of notation, we will denote the $n$-fold sum as $a(n)$ such that $a$ is 
        added to itself $n$ times i.e. $a(0) = 0$, $a(1) = a$, $a(2) = a + a$, etc.
        \spacing

        We'll then assume for $n \in \N$,
        $$a(n) = n \cdot a$$
        We will show that for $S(n) \in \N$,
        $$a(S(n)) = S(n) \cdot a$$
        using the fact that $n + 1$-fold sum of $a$ is equal to $a + $ n-fold sum of $a$.
        \spacing

        \textbf{Proof:} We will prove the proposition by using induction and inducting on $n$.
        Note that for notation purposes, we will denote the $n$-fold sum as $a(n)$ such that $a$ is 
        added to itself $n$ times i.e. $a(0) = 0$, $a(1) = a$, $a(2) = a + a$, etc.
        \spacing

        First, we'll show the base case $n = 0$ to be true.
        \begin{align*}
            a(n) \stackrel{?}{=} n \cdot a \\
            a(0) \stackrel{?}{=} 0 \cdot a \\
            0 \stackrel{?}{=} 0 \cdot a && \text{(the 0-fold sum of $a$ is just 0)} \\
            0 \stackrel{?}{=} a \cdot 0 && \text{(commutativity of multiplication)} \\
            0 = 0 && \text{(definition of multiplication)} \\
        \end{align*}
        Thus we have shown the base case $n = 0$ to be true by using axiom 1.
        \spacing

        Now, let's start with the indutive step and assume that for any $n \in \N$,
        $$a(n) = n \cdot a$$
        We will show that for $S(n) \in \N$,
        $$a(S(n)) = S(n) \cdot a$$
        Before we start working with the $S(n)$ statement, it's important to note that
        \begin{align}x + 1 = x + S(0) = S(x + 0) = S(x)\end{align}
        Now, we can look at the left side of the $S(n + 1)$ statement and show the following:
        \begin{align*}
            a(S(n)) \\
            a(n + 1) && \text{(by statement (2) above)} \\
            a + a(n) \\
            a + (n \cdot a) && \text{(inductive assumption)} \\
            a + (a \cdot n) && \text{(commutativity of multiplication)} \\
            a + 0 + (a \cdot n) && \text{(definition of addition)} \\
            a + (a \cdot 0) + (a \cdot n) && \text{(definition of multiplication)} \\
            a \cdot (S(0)) + (a \cdot n) && \text{(definition of multiplication)} \\
            a \cdot 1 + a \cdot n && \text{(definition of S(0))} \\
            a \cdot (1 + n) && \text{(distributivity theorem)} \\
            a \cdot (n + 1) && \text{(commutativity of addition)} \\
            a \cdot S(n) && \text{(by statement (2))} \\
            S(n) \cdot a && \text{(commutativity of multiplication)} \\
        \end{align*}
        Thus, we have proven the $S(n)$ statement using our inductive assumption.
        \spacing

        By induction, we now know that for $n \in \N$, the $n$-fold sum $a + \cdots + a = n \cdot a$. \qed
    \end{paragraph}

    \begin{paragraph}{3)}
        \textbf{a) Proposition:} Let $a, b, c \in \N$ and $a = b \cdot c$. Show that $b \leq a$.
        \spacing

        \textbf{Proof:} To prove that $b \leq a$, we'll start with the given that $a = b \cdot c$.
        From question 2, we know that $b \cdot c = b + \cdot b$ where $b + \cdot + b$ is the c-fold sum.
        Since $a$ is equal to sum of many $b$ parts, we can say that $a$ is greater than the sum of one of those
        $b$ parts. This is the same as stating that $b \leq a$. \qed
        \bigskip

        \textbf{b) Proposition:} Let $a \in \N$. Show that $a \leq a$.
        \spacing

        \textbf{Proof:} By the definition of addition, $a + 0 = a$. From this, it's easy to see that $a \leq a$. \qed
        \bigskip

        \textbf{c) Proposition:} $a \leq b \land b \leq c \Rightarrow a \leq c$
        \spacing

        \textbf{Proof:} We'll start by assuming that $a \leq b$ and $b \leq c$. We can rewrite them as 
        $a - b \leq 0$ and $b - c \leq 0$. Adding these two inequalities, we get $a - b + b - c \leq 0$.
        This can be simplified to $a - c \leq 0$ which is equivalent to $a \leq c$ \qed
        \bigskip

        \textbf{d) Proposition:} $a \leq b \land b \leq a \Rightarrow a = b$.
        \spacing

        \textbf{Proof:} Let there be a $c, d \in \N$ such that $a + c = b$ and $b + d = a$. Substituiting $b$
        from the first equation to the second, we get that $a + c + d = a$. This is equivalent to 
        $a + c + d = a + 0$. From what we proved in part (a), we can rewrite that as $c + d = 0$. Since 
        $c, d \in \N$, axiom 7 prevents $c$ or $d$ from being anything but 0. This can be shown through a proof of
        contradiction. Assume that $c \neq 0$. This implies that since $c \in n$, there's some $x \in \N$ such that
        $S(x) = c$. That means that $S(x) + d = 0$. This is the same as $d + S(x) = 0$ by the commutativity of addition,
        and axiom 7 prevents this statement from ever being true, so $c = 0$. This would mean that $a + c = b$
        can be rewritten as $a + 0 = b$ which is equivalent to $a = b$ by the definition of addition. Thus, 
        we have prove that if $a \leq b$ and $b \leq a$, then $a = b$. \qed
        \bigskip

        Full disclosure, I am the least confident on the last problem of this set. I tried to think
        of ways to work up to what needed to be proved using the axioms and induction, but after hours of head scratching
        and a lot of scratch paper, I had to rely heavily on intution and create what seems to me as 
        huge logical leaps from axioms to statements to supply the proofs. I'm looking forward
        to see how the proofs for these statements are formualted, and I would greatly appreciate any advice
        on how I can tackle a problem similar to this in the future.
    \end{paragraph}
\end{document}