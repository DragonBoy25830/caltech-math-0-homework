\documentclass[12.0pt, letterpaper]{article}
\usepackage[margin = 1in]{geometry}
\usepackage{amsthm}
\usepackage{amsmath}
\usepackage{amssymb}
\usepackage{fancyhdr}
\usepackage{pgfplots}
\pgfplotsset{compat=1.16}

\author{Sub}
\title{Test Homework}
\pagestyle{fancy}
\renewcommand{\headrulewidth}{0pt}

\newcommand{\biconditional}{\leftrightarrow}
\newcommand{\leftIndent}{\hspace*{8mm}}
\newcommand{\spacing}{\vspace*{6.0pt}}
\newcommand{\N}{\mathbb{N}}
\newcommand{\Z}{\mathbb{Z}}
\newcommand{\Q}{\mathbb{Q}}
\newcommand{\R}{\mathbb{R}}
\newcommand{\C}{\mathbb{C}}
\newcommand{\separate}{\makebox[0.5\columnwidth]{\dotfill}}

\fancyhf{}
\rhead{
    Subham Sahoo\\
    Teacher: Professor Pelayo\\
    Math 0\\
}
\rfoot{Page \thepage}

% Absolutely goated reference: https://oeis.org/wiki/List_of_LaTeX_mathematical_symbols

\begin{document}
    \begin{center}
        \huge\sc{Lesson 6 Solutions}
    \end{center}

    \begin{paragraph}{1)}
        \textbf{Proposition:} Let $r \neq 1$. Show that $\sum_{j=0}^n r^j = \frac{1-r^{n + 1}}{1-r}$.  
        \spacing    
        
        \textbf{Discussion:} Since our summation is indexed by real numbers and it seems that we 
        can relate $k$ to $k + 1$ using exponent properties, a proof of induction would be useful here.
        \spacing
        
        \begin{itemize}
            \item{
            Identify $A(n)$: $A(n)$ is the statement that $\sum_{j=0}^n r^j = \frac{1 - r^{n + 1}}{1 - r}$
            }
            
            \item{
            Base Case: We'll show that A(0) is true which makes sense because the summation
            of $r^0$ is just equal to $1$
            }
            
            \item{
            Inductive Step: We'll assume that for $k \in \N$, $$\sum_{j=0}^{k}r^j = \frac{1 - r^{k+1}}{1 - r}$$ We will show that $A(k + 1)$ is also true by showing that $$\sum_{j=0}^{k + 1} r^j = \frac{1-r^{(k + 1) + 1}}{1-r}$$
            }
        \end{itemize}
        \spacing

        \textbf{Proof:} We will use a proof by induction to prove the statement $A(n)$ given by 
        $\sum_{j=0}^{n} r^j = \frac{1 - r^{n+1}}{1-r}$ for all $n \in \N$.
        \spacing
        
        First, we'll prove the base case $A(0)$ to be true. Computing $A(0)$, we get
        $$\sum_{j=0}^{0} r^j= \frac{1 - r^{0+1}}{1-r} = \frac{1-r}{1-r} = 1$$
        Since $r^0 = 1$, we can see that $A(0)$ is true.
        \spacing
        
        Now, we'll start with the inductive step. We'll assume for $k \in \N$, $A(k)$ is true.
        That is,
        $$\sum_{j=0}^{k} r^j = \frac{1 - r^{k+1}}{1-r}$$
        We will show that $A(k + 1)$ is true by showing that 
        $$\sum_{j=0}^{k + 1} r^j = \frac{1 - r^{(k + 1)} + 1}{1 - r}$$
        Looking at the left side of the $A(k + 1)$ statement, we can use our inductive
        assumption to show the following:
        $$\sum_{j=0}^{k+1} r^j = (\sum_{j=0}^{k} r^j) + r^{k+1} = (\frac{1 - r^{k+1}}{1-r}) + r^{k+1} = \frac{1 - r^{k+1}}{1-r} + \frac{r^{k+1}(1-r)}{1-r} = \frac{1-r^{k+1} + r^{k+1} - r\cdot r^{k+1}}{1-r} =$$
        $$\frac{1 - r^{(k + 1) + 1}}{1-r}$$
        Thus, we have used our inductive assumption to show that $A(k+1)$ is true.
        \spacing
        
        By induction, we now know that the statement $A(n)$ given by $\sum_{j = 0}^{n} = \frac{1 - r^{n+1}}{1-r}$ is true for all $n \in \N$. \\
        \proofEnd
        
    \end{paragraph}

    \bigskip    

    \begin{paragraph}{2)}
        Consider $f(x) = \frac{1}{1-x}$
        \spacing

        \textbf{a)} Computer the first several derivatives of $f$ and use them to conjecture a pattern for $f^{(n)}(x)$.\\
        \separate
        \begin{center}
            \begin{tabular}{||c | c||}
                \hline
                $n$ & $f^{(n)}(x)$ \\ [0.5ex]
                \hline\hline
                0 & $(1 - x)^{-1}$ \\
                \hline
                1 & $-(1 - x)^{-2}$ \\
                \hline
                2 & $2(1 - x)^{-3}$ \\
                \hline
                3 & $-6(1 - x)^{-4}$ \\
                \hline
                4 & $24(1 - x)^{-5}$ \\
                \hline
            \end{tabular}
            \spacing

            Conjecture: $f^{(n)}(x) = (-1)^n n!(1-x)^{-(n + 1)}$
        \end{center}
        
        \bigskip
        
        \textbf{b) Proposition:} Let $x \neq 1$. Show that $f^{(n)}(x) = (-1)^n n!(1 - x)^{-(n + 1)}$ for $n \in \N$
        \spacing
        
        \textbf{Discussion:} Since the proposition is indexed by natural numbers and it seems that
        we can use factorial and exponent properties to relate $k$ to $k + 1$, a proof by induction would work well 
        here.
        \begin{itemize}
            \item{
                Identify $A(n)$: $A(n)$ is the statement that $f^{(n)}(x) = (-1)^n n!(1 - x)^{-(n + 1)}$
            }

            \item{
                Base Case: We'll show that $A(0)$ is true by using the fact that the 0-th derivative of a function is just the function itself
            }

            \item{
                Inductive Step: We'll assume that for some $k \in \N$,
                $$f^{(k)}(x) = (-1)^k k! (1 - x)^{-(k + 1)}$$
                We will show that $A(k + 1)$ is true by showing that 
                $$f^{(k + 1)}(x) = (-1)^{(k + 1)} (k + 1)! (1-x)^{-((k + 1) + 1)}$$
            }
        \end{itemize}
        \spacing
        
        \textbf{Proof:} We will used a proof by induction to prove the statement $A(n)$ given by \\$f^{(n)} = (-1)^n n! (1 - x)^{-(n + 1)}$ for $n \in \N$.
        \spacing

        First, we'll show the base case $A(0)$ to be true. Computing $A(0)$, we get
        $$f^0(x) = (-1)^0 0! (1 - x)^{-(0 + 1)}$$
        This is equivalent to $f(x) = \frac{1}{1-x}$ which means $A(0)$ is true.
        \spacing

        Now, we'll start with the inductive step by assuming that for some $k \in \N$, $A(k)$ is true. That is,
        $$f^{(k)}(x) = (-1)^k k! (1 - x)^{-(k + 1)}$$

        We will show that $A(k + 1)$ is true by showing that 
        $$f^{(k + 1)}(x) = (-1)^{(k + 1)} (k + 1)! (1-x)^{-((k + 1) + 1)}$$

        Looking at the left hand side of the $A(k + 1)$ statement, we can use our inductive assumption to show the following
        
        \begin{align*}
            f^{(k + 1)}(x) = \frac{d}{dx}f^{(k)}(x) & = \frac{d}{dx} (-1)^k k! (1 - x)^{-(k + 1)}\\
            & = (-1)^k k! \frac{d}{dx} (1 - x)^{-(k + 1)} \\
            & = (-1)^k k! \cdot -(k + 1) (1-x)^{-(k + 1) - 1} \\
            & = (-1)(-1)^k (k + 1)k! (1 - x)^{-((k + 1) + 1)} \\
            & = (-1)^{k + 1} (k + 1)! (1 - x)^{-((k + 1) +1)}
        \end{align*}
        Thus, we have used our inductive assumption to show that $A(k + 1)$ is true.
        \spacing
        
        By induction, we now know that the statement $A(n)$ given by $f^{(n)}(x) = (-1)^n n! (1 - x)^{-(n + 1)}$ is true for all $n \in \N$.\\
        \proofEnd
    \end{paragraph}
    
    \bigskip

    \begin{paragraph}{3)}
        \textbf{Proposition:} Let $x > -1$. Show that $(1 + x)^n \geq 1 + nx$ where $n \in \{m \in \N \mid m \geq 1\}$
        \spacing
        
        \textbf{Discussion:} Since the inequality can be said to be indexed by natural numbers and it seems that we can use the combination of inequalities and exponent properties to relate $k$ to $k + 1$.
        
        \begin{itemize}
            \item {
            Identify $A(n)$: $A(n)$ is the statement $(1 + x)^n \geq 1 + nx$
            }
            
            \item{
            Base Case: We will show that A(1) is true because some quantity $a$ will always be greater than or equal to $a$
            }
            
            \item{
            Inductive step: We'll assume that for $k \geq 1$ such that $k \in \N$,
            $$(1 + x)^k = 1 + kx$$
            We will show that $A(k + 1)$ is also true by showing that 
            $$(1 + x)^{k + 1} = 1 + (k + 1)x$$
            }
        \end{itemize}
        \spacing
        
        \textbf{Proof:} We will use a proof by induction to prove the statement $A(n)$ given by 
        $(1 + x)^n \geq 1 + nx$ where $x > -1$ and $n$ is all the integers such that $n \geq 1$.
        \spacing

        First we'll prove the base case $A(1)$ to be true. Computing $A(1)$, we get
        $$(1 + x) ^ 1 \stackrel{?}{=} 1 + (1)x$$
        $$1 + x \geq 1 + x$$
        Since we created a logically true statement, we can see that $A(1)$ is true.
        \spacing

        Now, we'll start with the inductive step by assuming that for $k \geq 1$ such that $k \in \N$, $A(k)$ is true, that is,
        $$(1 + x)^k \geq 1 + kx$$
        We will show that $A(k +1)$ is true by showing that 
        $$(1 + x)^{k +1} \geq 1 + (k +1)x$$
        Looking at the left side of the $A(k + 1)$ statement, we can use our inductive assumption to show the following:
        $$(1 + x)^{k + 1}$$
        $$(1 + x)^k \cdot (1 + x)$$
        $$(1 + x)^k \cdot (1 + x) \geq (1 + kx) \cdot (1 + x)$$
        $$(1 + x)^k \cdot (1 + x) \geq 1 + x + kx + kx^2$$
        $$(1 + x)^k \cdot (1 + x) \geq 1 + x + kx + kx^2 \geq 1 + x + kx$$
        $$(1 + x)^k \cdot (1 + x) \geq 1 + x + kx$$
        $$(1 + x)^{k + 1} \geq 1 + (k + 1)x$$
        Thus, we have used our inductive assumption to show that $A(k + 1)$ is true.
        \spacing
        
        By induction, we now know that the statement $A(n)$ given by $(1 + x)^n \geq 1 + nx$ where 
        $x > -1$ is true for all $n \geq 1$ such that $n \in \N$.\\
        \proofEnd
    \end{paragraph}
\end{document}