\documentclass[10pt]{report}
\makeindex

\usepackage{amsmath, amsfonts, amssymb, amstext, amscd, amsthm, makeidx, graphicx, hyperref, url}
\allowdisplaybreaks

\setcounter{chapter}{0}
\topmargin -.5in
\textheight 9.0in

\begin{document}

\begin{center}\textsc{\Large Transition to Mathematical Proofs}

\textsc{\large Chapter 5 - Complex Numbers Assignment}

\bigskip

\end{center}

\noindent\textsc{Instructions:}  For the below questions, show all of your work.  For the proofs, be sure that you \\

\noindent
(i) write a complete proof in full English sentences; \\
(ii) if hand-writing, write legibly and clearly.

\medskip



\noindent\textsc{Note:}  Discussion sections are no longer required.  You may, of course, include them in your assignments, as they may help the grader give more helpful feedback.  


\bigskip

\noindent\textbf{Question 1.}  Similar to how we obtained the double-angle formulae in the notes, use the Euler equation to show the two angle-sum formulae hold:  $$\sin (\alpha+ \beta) = \sin \alpha \cos \beta + \sin \beta \cos \alpha; $$ $$\cos(\alpha + \beta) = \cos \alpha \cos \beta - \sin \alpha \sin \beta.$$

\bigskip


\noindent\textbf{Question 2.}  

\begin{itemize}

\item[(a)]  Show that $|z| = \textrm{Re}(z)$ if and only if $z$ is a non-negative real number.

\item[(b)]  Show that $\left(\overline z\right)^2 = z^2$ if and only if $z$ is purely real or purely imaginary (i.e., its real part is $0$).  

\end{itemize}

\bigskip

\noindent\textbf{Question 3.}  The modulus of a complex number is, in many ways, a generalization of the absolute value of a real number.  Here, we give another property of the modulus that the absolute value of a real number already enjoys.  \\ \medskip
 If $z, w \in \mathbb C$, show that $$|z \cdot w| = |z| \cdot |w|$$ in the following two ways:

\begin{itemize}

\item[(a)]  By using the Cartesian form $z = a+bi$ and $w = c+di$ for the complex numbers $z$ and $w$.

\item[(b)]  By using the polar form $z = r_1e^{i\theta_1}$ and $w = r_2e^{i \theta_2}$ for the complex numbers $z$ and $w$.  
\end{itemize}

\bigskip

\noindent\textbf{Question 4.}  Below, we will prove a remarkable fact about real polynomials using complex numbers.  For the parts below, let $z = a+bi$ and $w = c+di$ be complex numbers. 

\begin{itemize}

\item[(a)]  Show that $\overline{z+w} = \overline z + \overline w$. 

\item[(b)]  Show that $\overline{z \cdot w} = \overline z \cdot \overline w$.   

\item[(c)]  Use (b) to show that $\overline{z^n} = \left(\overline z\right)^n$ for any natural number $n \in \mathbb N$.  

\item[(d)]  Consider the following polynomial $p(z)$ with \emph{real coefficients}:  $$p(z) = \alpha_nz^n + \alpha_{n-1}z^{n-1} + \cdots + \alpha_1 z + \alpha_0,$$ where each $\alpha_i$ is a real number.  Show that if a complex number $w$ is a root to the above polynomial with real coefficients, then its conjugate $\overline w$ is also a root to the same polynomial.  That is, use (a) - (c) to show that if $p(w) = 0$, then $p(\overline w) = 0$.  

\end{itemize}


\end{document}