\documentclass[12.0pt, letterpaper]{article}
\usepackage[margin = 1in]{geometry}
\usepackage{amsmath}
\usepackage{amssymb}
\usepackage{fancyhdr}
\usepackage{pgfplots}
\pgfplotsset{compat=1.16}

\author{Sub}
\title{Test Homework}
\pagestyle{fancy}
\renewcommand{\headrulewidth}{0pt}

\fancyhf{}
\rhead{
    Subham Sahoo\\
    Teacher: Mr. Wit\\
    Section 123\\
}
\rfoot{Page \thepage}

\begin{document}
    \begin{paragraph}{1.1)}
        \textit{
            \textbf{A)} Find the sum of $34$ and $126$ using a calculator.
        }

        \begin{equation*}
            34 + 126 = \framebox{160}
        \end{equation*}

        \textit{
            \textbf{B)} Find the sum using long additon.
        }
        
        \begin{center}
            \begin{tabular}{cccc}
                &  & 1 &   \\
                &  & 3 & 4 \\
            +  & 1 & 2 & 6 \\
                \hline
               & 1 & 6 & 0 \\
                
            \end{tabular}
        \end{center}
    \end{paragraph}
    
    \begin{paragraph}{1.2)}
        \textit{
            Evaluate the following definite integral:
        }

        \begin{equation*}
            \int_{0}^{3} \frac{2x}{\sqrt{x^2+4}} dx 
        \end{equation*}
        
        $$
            u = x^2 + 4 
            \quad
            du = 2xdx
        $$
        
        \begin{equation*}
            \Rightarrow\quad
            \int_{0^2+4}^{3^2+4} \frac{1}{\sqrt{u}} du = 
            \left. 2\sqrt{u} \right |_{4}^{13}
        \end{equation*}
        
        \begin{equation*}
            2\left(\sqrt{13} - \sqrt{4}\right) \approx \framebox{3.211}
        \end{equation*}
    \end{paragraph}

    \begin{paragraph}{2.1)}
        \textit{
            \textbf{A)} Find the roots of the quadratic equation
            $y = x^2+2x-3$
        }

        \begin{equation*}
            x = \frac{-b\pm\sqrt{b^2-4ac}}{2a}
            = \frac{-2\pm\sqrt{2^2-4(-1)(-3)}}{2(1)}
            = \frac{-2\pm\sqrt{16}}{2}
        \end{equation*} 
        \begin{center}
            \framebox{$x = 1 \quad \&\ \quad -3$}
        \end{center}

        \textit{
            \textbf{B)} Graph the same function to verify those points.
        }

        \begin{center}
            \begin{tikzpicture}
                \begin{axis} [
                    xlabel = $x$,
                    ylabel = $y$, 
                    xmin = -4, xmax = 2, ymin = -5, ymax = 5
                ]
                  \addplot[mark = none]{x^2+2*x-3};
                  \addplot[mark=*, red, only marks] coordinates{(-3, 0)(1, 0)};
                \end{axis}
            \end{tikzpicture}
        \end{center}

    \end{paragraph}
    \pagebreak
    \begin{paragraph}{2.2}
        \textit{A particle's location is $(1, 4, 7)$ at $t-0$,
        and it's velocity is given by $\vec{v}(t) = (4t+3)\hat{i} +
        (2t)\hat{j} + (6t+1)\hat{k}$.Find the particle's location
        as a function of time, and evaluate for $t=6$ }

        \textbf{A)} 

        \begin{equation*}
            \vec{x}(t) = (2t^2+3t+x_i)\hat{i} + (t^2+c_j)\hat{j} + 
            (3t^2 + t + x_k)\hat{k}
        \end{equation*}
        \begin{equation*}
            c_i = 1 \quad
            c_j = 4 \quad
            c_k = 7 
        \end{equation*}
        \begin{center}
            \framebox{$\vec{x}(t) = (2t^2+3t+1)\hat{i} + (t^2 + 4)\hat{j} + 
            (3t^2 + t + 7)\hat{k}$}
        \end{center}

        \textbf{B)}
        \begin{equation*}
            \vec{x}(6) = 91\hat{i} + 40\hat{j} + 121\hat{k}
        \end{equation*}
    \end{paragraph}

\end{document}